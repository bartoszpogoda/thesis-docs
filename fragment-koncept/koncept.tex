\documentclass[a4paper,11pt]{article}

\usepackage{latexsym}
\usepackage{graphicx}
\usepackage{float}
\usepackage[margin=2cm]{geometry}
\usepackage{lscape}
\usepackage{underscore}
\usepackage{amsmath}
\usepackage{blindtext}
\usepackage{listings}
\usepackage{xcolor}
\usepackage{mathtools}

\usepackage[all]{nowidow}

\usepackage[LY1]{fontenc}
\usepackage[utf8x]{inputenc}
\usepackage{polski}
\usepackage[lf,enc=t1]{berenis}

\DeclareTextCompositeCommand{\k}{LY1}{a}
  {\oalign{a\crcr\noalign{\kern-.27ex}\hidewidth\char7}}


\lstset{
    language=C++,
    frame=tb, % draw a frame at the top and bottom of the code block
    tabsize=4, % tab space width
    showstringspaces=false, % don't mark spaces in strings
    numbers=left, % display line numbers on the left
    backgroundcolor=\color{black!5}, 
   % commentstyle=\color{green}, % comment color
    keywordstyle=\color{blue}, % keyword color
    stringstyle=\color{red}, % string color  
    inputencoding=utf8x,
    extendedchars=\true,
    literate={ą}{{\k{a}}}1
             {Ą}{{\k{A}}}1
             {ę}{{\k{e}}}1
             {Ę}{{\k{E}}}1
             {ó}{{\'o}}1
             {Ó}{{\'O}}1
             {ś}{{\'s}}1
             {Ś}{{\'S}}1
             {ł}{{\l{}}}1
             {Ł}{{\L{}}}1
             {ż}{{\.z}}1
             {Ż}{{\.Z}}1
             {ź}{{\'z}}1
             {Ź}{{\'Z}}1
             {ć}{{\'c}}1
             {Ć}{{\'C}}1
             {ń}{{\'n}}1
             {Ń}{{\'N}}1
	     {*}{\normalfont{*}}1,
}


\begin{document}

\tableofcontents
\newpage

\section{Koncept systemu}

\subsection{Sugerowanie przeciwników}

Jednym z podstawowych celów systemu jest wspomaganie procesu poszukiwania rywali do gry. Warto zaznaczyć, że osobą podejmującą decyzję o wyzwaniu drużyny na pojedynek jest kapitan drużyny (w zgodzie z zawodnikami), system nigdy nie podejmuje decyzji samodzielnie - jego funkcją jest zasugerowanie kapitanowi przeciwników oraz "wyjaśnienie" każdej z sugestii.

\subsection{Podstawowe kryterium dopasowywania}

Podstawowym celem dopasowywania przeciwników jest maksymalizacja satysfakcji z gry drużyn oraz maksymalizacja szans na umówienie spotkania.

\subsection{Satysfakcja - herustyki}

Satysfakcja z gry jest pojęciem niemożliwym do zmierzenia. Dlatego w celu umożliwienia oceny potencjalnej satysfakcji z rozgrywki dwóch drużyn zostały zdefiniowane następujące przesłanki, z których może wynikać satysfakcja z gry:

\begin{enumerate}
\item przybliżony poziom umiejętności drużyn,
\item przybliżony wiek zawodników,
\item dobre wspomnienia po poprzednich rozgrywkach,
\item poziom fair play drużyny przeciwnej\ldots
\end{enumerate}


\subsubsection{Poziom umiejętności drużyn - rezygnacja z tego raczej}

Kluczowym czynnikiem wpływającym na satysfakcję płynącą z rozgrywki jest różnica umiejętności drużyn. Gra ze znacznie lepszymi bądź dużo gorszymi przeciwnikami może prowadzić do znudzenia drużyny lepszej, bądź zniechęcenia drużyny gorszej. Oczywiście w przypadku, gdy obydwie drużyny są bardzo przyjaźnie nastawione, spotkanie takie może prowadzić do dobrych efektów t.j. przekazania wiedzy. <To trzeba lepiej ubrać w słowa, można zaznaczyć że opcjonalny moduł będzie sugerował drużyny wyłacznie na podstawie wyznaczonego poziomu umiejętności>.



\subsection{Czynniki wpływające na szansę umówienia spotkania}

Przesyłanki dotyczące szansy na umówienie spotkania:

\begin{enumerate}
\item pokrywajace się zadeklarowane godziny możliwości gry,
\item aktywność obydwu drużyn\ldots
\end{enumerate}









\end{document}