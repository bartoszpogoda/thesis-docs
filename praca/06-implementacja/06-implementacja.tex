\chapter{Implementacja i działanie systemu Team Challenge}

\section{Środowisko implementacji}

Implementacja systemu odbywała się przy użyciu komputera wyposażonego w 8GB pamięci fizycznej oraz cztero-rdzeniowy procesor Intel Core i5-6300HQ (taktowanie 2.30GHz). Sprzęt o przytoczonych parametrach okazał się w pełni wystarczający dla przebiegu implementacji oraz lokalnego uruchamiania serwera aplikacji. Systemem operacyjnym używanym przy implementacji był Windows w wersji 10 Education. Wszystkie użyte programy i narzędzia dobrze współpracują z tym systemem.

\subsection{Wykorzystane narzędzia}


\begin{table}[H]
\centering\small
\caption{Zestawienie narzędzi wykorzystywanych podczas implementacji systemu}
\label{tab:szablon}
\begin{tabularx}{\linewidth}{|p{.2\linewidth}|p{.1\linewidth}|p{.1\linewidth}|X|}\hline
Nazwa programu & Wersja & Producent & Cel\\ \hline\hline

IntelliJ IDEA & 2018.2 & Jetbrains & Implementacja aplikacji serwerowej (Java) \\ \hline

Webstorm & 2018.2 & Jetbrains & Implementacja aplikacji klienckiej (Angular) \\ \hline

Postman & 6.5.2 & Postman & Testowanie końcówek RESTowych aplikacji serwerowej \\ \hline

Google Chrome & 70 & Google & Testowanie aplikacji klienckiej \\ \hline

Git & 2.18.0 & - & Kontrola wersji \\ \hline

\end{tabularx}
\end{table}

\section{Przebieg implementacji aplikacji serwerowej}

\subsection{Struktura projektu}

Struktura katalogow 

\subsection{Warstwy aplikacji}

DTO, RestController, Service, Dao

\subsection{Metody dostępu do danych}

JDBC Hibernate JPA (SPring Data JPA)

JPA, Paginacja dla dużych zbiorów danych, Specification

\subsection{Bezpieczeństwo}

Przechowywanie haseł, Autoryzacja zapytań

\subsection{Algorytm poszukiwania rywali}

\section{Przebieg implementacji aplikacji klienckiej}

Zarządzanie stanem aplikacji Obsluga stanu aplikacji

\subsection{Środowisko}

Na jakim komputerze, jaki program itp. Testy na Chrome itp
