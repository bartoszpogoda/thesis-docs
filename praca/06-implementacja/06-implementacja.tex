\chapter{Implementacja i działanie systemu Team Challenge}

\section{Środowisko implementacji}

Implementacja systemu odbywała się przy użyciu komputera wyposażonego w 8GB pamięci fizycznej oraz cztero-rdzeniowy procesor Intel Core i5-6300HQ (taktowanie 2.30GHz). Sprzęt o przytoczonych parametrach okazał się w pełni wystarczający dla przebiegu implementacji oraz lokalnego uruchamiania serwera aplikacji. Systemem operacyjnym używanym przy implementacji był Windows w wersji 10 Education. Wszystkie użyte programy i narzędzia dobrze współpracują z tym systemem.

\subsection{Wykorzystane narzędzia}


\begin{table}[H]
\centering\small
\caption{Zestawienie narzędzi wykorzystywanych podczas implementacji systemu}
\label{tab:szablon}
\begin{tabularx}{\linewidth}{|p{.2\linewidth}|p{.1\linewidth}|p{.1\linewidth}|X|}\hline
Nazwa programu & Wersja & Producent & Cel\\ \hline\hline

IntelliJ IDEA & 2018.2 & Jetbrains & Implementacja aplikacji serwerowej (Java) \\ \hline

Webstorm & 2018.2 & Jetbrains & Implementacja aplikacji klienckiej (Angular) \\ \hline

Postman & 6.5.2 & Postman & Testowanie końcówek RESTowych aplikacji serwerowej \\ \hline

Google Chrome & 70 & Google & Testowanie aplikacji klienckiej \\ \hline

Git & 2.18.0 & - & Kontrola wersji \\ \hline

\end{tabularx}
\end{table}

\section{Implementacja aplikacji serwerowej}

\subsection{Struktura projektu}

Podstawowy szkielet projektu został utworzony przy użyciu Spring Initializr. Jako narzędzie służące do zarządzania zależnościami oraz budowy projektu wybrany został Apache Maven. 

\begin{comment}
W tabeli zostały przedstawione wykorzystane moduły Springa oraz inne biblioteki. 
\end{comment}

  Klasy tworzące aplikację serwerową zostały podzielone na pakiety pod względem tematycznym, co zaprezentowano na rysunku~\ref{fig:packages}. Podział taki pozwala na utrzymanie płaskiej struktury katalogów, a w związku z tym umożliwia szybkie odnajdywanie pożądanych plików.
  
  
\begin{figure}[ht]
\centering
\includegraphics[width=0.5\linewidth]{06-implementacja/rys/package-team.PNG}
\caption{Fragment struktury pakietów}
\label{fig:packages}
\end{figure}

\subsection{Architektura wielowarstwowa}

Podczas projektowania architektury aplikacji ważne jest aby była ona przejrzysta oraz otwarta na rozszerzanie. Jedną z cech, którą powinny posiadać komponenty dobrze zaprojektowanego oprogramowania zorientowanego obiektowo jest ograniczenie odpowiedzialności. Jedną z metod rozdzielania odpowiedzialności jest wyszczególnienie w projekcie warstw komponentów. Aplikacja serwerowa będąca przedmiotem niniejszej pracy została podzielona na warstwy zgodnie ze standardami zdefiniowanymi dla szkieletu Spring. Wyszczególnione warstwy wraz z kierunkami komunikacji zostały przedstawione na rysunku~\ref{fig:app-layers}.


\begin{comment}

Nawiazanie do: 
Understanding Spring Web Application Architecture: The Classic Way
Petri Kainulainen October 19, 2014
https://www.petrikainulainen.net/software-development/design/understanding-spring-web-application-architecture-the-classic-way/

\end{comment}

\begin{figure}[H]
\centering
\includegraphics[width=0.5\linewidth]{06-implementacja/rys/layers.PNG}
\caption{Warstwy aplikacji serwerowej}
\label{fig:app-layers}
\end{figure}


\subsection{Warstwa repozytoriów}

Repozytoria w frameworku Spring stanowią mechanizm dostępu do danych. Warstwa ta jest abstrakcją ukrywającą przed programistą szczegóły komunikacji z bazą danych takie jak: nawiązywanie oraz utrzymywanie połączenia, konstrukcja i wykonywanie zapytań, mapowanie wyników. Deklaracja fizycznych powiązań między aplikacją a bazą danych odbywa się za pomocą adnotacji. Na listingu~\ref{list:entity} przedstawiono powiązanie encji Team z fizyczną tabelą o nazwie Teams oraz mapowania przykładowych kolumn.

\begin{lstlisting}[label=list:entity, caption=Fragment przykładowej encji, basicstyle=\footnotesize\ttfamily]
@Entity
@Table(name = "Teams")
@Data
@Builder
public class Team {

    @Id
    @GeneratedValue(strategy = GenerationType.IDENTITY)
    @Column(name = "TeamID")
    private String id;

    @OneToOne(fetch = FetchType.EAGER)
    @JoinColumn(name = "ManagerID")
    private Player manager;

    @OneToMany(mappedBy = "team", cascade = CascadeType.ALL)
    private List<Player> players;
    
    // other fields with their mappings...
}
\end{lstlisting}

Dostęp do danych zapewniają repozytoria, czyli interfejsy oznaczone adnotacją @Repository. Wygodne jest rozszerzanie dostarczonego przez moduł Spring Data interfejsu CrudRepository, który definiuje podstawowe metody dostępu takie jak dodawanie, czytanie, modyfikacja oraz usuwanie encji. Pozostałe metody potrzebne do realizacji logiki biznesowej można definiować poprzez tworzenie metod o nazwach specyfikujących ich działanie. Fizyczne zapytania do bazy danych wyznaczane są na podstawie nazwy metody. Przykładową definicję repozytorium przedstawiono na listingu~\ref{list:repository}.

\begin{lstlisting}[label=list:repository, caption=Definicja przykładowego repozytorium, basicstyle=\footnotesize\ttfamily]
@Repository
public interface TeamRepository 
extends CrudRepository<Team, String>, JpaSpecificationExecutor<Team> {

    Optional<Team> findById(String id);

    List<Team> findByRegionIdAndDisciplineIdAndActiveIsTrue(String regionId,
     String disciplineId);
}
\end{lstlisting}

Dostarczenie interfejsu repozytorium rozszerzającego JpaSpecificationExecutor pozwala na wykonywanie bardziej zaawansowanych zapytań. Mechanizm ten został zastosowany przy tworzeniu zapytań, gdzie lista predykatów była ustalana w zależności od parametrów zapytania HTTP w sposób dynamiczny. Do budowy kwerend użyto klas Specification oraz Predicate pochodzących z modułu Sring Data JPA.


\subsection{Warstwa serwisów}

Podstawowym zadaniem serwisów jest przetwarzanie danych zgodnie z regułami biznesowymi. W zaimplementowanym systemie na poziomie serwisów również sprawdzane są prawa dostępu do zasobów. W Springu serwisy oznaczane są adnotacje @Service. Framework sam zajmuje się tworzeniem instancji serwisów, które mogą być użyte w pozostałych komponentach systemu. Przykładowy serwis został przedstawiony na listingu~\ref{list:service}.

\begin{comment}
Możę o wstrzykiwaniu zależności tutaj.
\end{comment}

\begin{minipage}{\linewidth}
\begin{lstlisting}[label=list:service, caption=Fragment przykładowego serwisu, basicstyle=\footnotesize\ttfamily]

@Service
public class TeamService {

    private TeamRepository teamRepository;
    private PositionService positionService;
    // other fields...
    
    @Transactional
    public Position setHome(String id, PositionDto positionDto) 
    	throws TeamNotFoundException, AccessForbiddenException {
    
        Team team = teamRepository.findById(id)
        .orElseThrow(TeamNotFoundException::new);
        
        if(!isManagedByCurrentUser(team)) {
            throw new AccessForbiddenException();
        }

        Position position = this.positionService.save(positionDto);
        team.setHome(position);

        return position;
    }
    
    // other methods...
}
\end{lstlisting}
\end{minipage}


\subsection{Warstwa zasobów}

Warstwa zasobów jest szczególna, z tego względu, że stanowi interfejs sytemu dla świata zewnętrznego. W ramach tej warstwy działa dostarczony przez szkielet Spring Dispatcher Servlet. Jest to komponent obsługujący żądania HTTP przychodzące do aplikacji. W ramach obsługi żądania są one przekazywane do konkretnych "kontrolerów", wybranych na podstawie URL żądania. Listing~\ref{list:resource}. przedstawia rejestracje kontrolera za pomocą adnotacji @RestController oraz @RequestMapping. Dispatcher Servlet w przypadku tak skonfigurowanej klasy będzie kierował zapytania o adresie /teams do kontrolera TeamResource. Zapytanie /teams/4  zostanie przekazane konkretnie do metody getTeam z argumentem wywołania równym 4.

Na poziomie warstwy zasobów odbywa się również walidacja przychodzących danych.

\begin{minipage}{\linewidth}
\begin{lstlisting}[label=list:resource, caption=Przykładowa rejestracja kontrolera, basicstyle=\footnotesize\ttfamily]

@RestController
@RequestMapping("/teams")
public class TeamResource {

    private TeamService teamService;

    private DtoMappingService mappingService;

    @GetMapping("/{id}")
    public ResponseEntity<TeamDto> getTeam(@PathVariable String id)
     throws ApiException {
        return teamService.findById(id)
                .map(mappingService::mapToDto)
                .map(ResponseEntity::ok)
                .orElseThrow(TeamNotFoundException::new);
    }
    
    // other methods ...
}

\end{lstlisting}
\end{minipage}

Pełna specyfikacja REST API została przedstawiona w załączniku X.

\subsection{Algorytm poszukiwania rywali}

Algorytm poszukiwania rywali został zaimplementowany w oparciu o zagadnienie optymalizacji wielokryterialnej, które zostało przybliżone w trzecim rozdziale niniejszej pracy. Podczas prac nad algorytmem uwzględniono domenę problemu oraz charakter danych, na jakich będzie on operował. W związku z tym tradycyjny schemat oceny decyzji został rozszerzony, co zostało opisane w dalszej części tego rozdziału. Rysunek~\ref{fig:diagram-alg-ext} przedstawia uproszczony schemat działania algorytmu zrealizowanego w systemie.

\begin{figure}[H]
\centering
\includegraphics[width=0.8\linewidth]{03-koncept/rys/algorytm.PNG}
\caption{Schemat działania algorytmu oceny decyzji}
\label{fig:diagram-alg-ext}
\end{figure}

\subsubsection{Kryteria}

W systemie zostały wyszczególnione dwa główne rodzaje kryteriów - numeryczne oraz logiczne. Kryteria numeryczne obejmują kryteria, które da się zmierzyć i wyrazić w postaci liczbowej. Uwzględniono tutaj takie kryteria jak: różnica wieku, różnica umiejętności oraz odległość między drużynami. Kryteria logiczne wyrażają dodatkowe wskaźniki, które mają wpływ na dobre dopasowanie drużyn, jednak nie są policzalne.  Kryteriom tym przypisane są wartości logiczne, które oznaczają czy dane kryterium zostało spełnione. Przykładowo wartość logiczna kryterium dotyczącego poziomu fair play będzie ustawiona jeżeli średnia ocen fair play potencjalnych rywali jest większa lub równa 4 (maksymalnie 5).

Dodatkowo wyszczególniono generyczną klasę, która opakowuje dowolne kryterium dodając informację o znormalizowanej wartości liczbowej. 

Zastosowaną hierarchię klas przedstawiono na rysunku ref. Rozbudowa funkcjonalności o nowe kryteria sprowadza się do utworzenia dodatkowej implementacji w odpowiednim miejscu hierarchii.  

\begin{figure}[H]
\centering
\includegraphics[width=\linewidth]{06-implementacja/rys/criterion-package-classes.PNG}
\caption{Hierarchia klas: kryteria}
\label{fig:criterion-classes}
\end{figure}

Kryteria dla poszczególnych decyzji generowane są na podstawie zgromadzonych danych o drużynach, zawodnikach oraz wynikach spotkań. Przykładową metodę przedstawiono na listingu XX. 


\begin{minipage}{\linewidth}
\begin{lstlisting}[label=list:age-crit-gen, caption=Przykładowa rejestracja kontrolera, basicstyle=\footnotesize\ttfamily]
public AgeCriterion ageCriteria(Team hostTeam, Team otherTeam) {
  double averageAgeHostTeam = hostTeam.getPlayers().stream()
                .mapToDouble(playerService::getAge).average().orElse(0);
  double averageAgeOtherTeam = otherTeam.getPlayers().stream()
                .mapToDouble(playerService::getAge).average().orElse(0);

  return new AgeCriterion(
    averageAgeOtherTeam - averageAgeHostTeam, 
    averageAgeHostTeam, 
    averageAgeOtherTeam
  );
}
\end{lstlisting}
\end{minipage}

\subsubsection{Normalizacja kryteriów}

 Normalizacja poszczególnych kryteriów przebiega przy użyciu różnych funkcji, dopasowanych do charakteru danych. Przykładem uzasadniającym konieczność zastosowania takiej modyfikacji może być porównanie dwóch kryteriów: odległości drużyn oraz średniego wieku zawodników. O ile kryterium odległości może być normalizowane w pełni liniowo, o tyle dla różnicy wieku taka metoda normalizacji jest błędna. Różnica wieku między dwoma zawodnikami, którzy mają 15 oraz 20 lat jest znacznie bardziej istotna aniżeli różnica między zawodnikami w wieku 30 oraz 35 lat - nie można tutaj zastosować operatora w pełni liniowego. Dodatkowo w domenie problemu wyróżniono kryteria nieliczbowe - cechy drużyny, które mogą mieć duży wpływ na jakość dopasowania, np. zadeklarowana chęć ponownej gry z daną drużyną.  
 
O metodach normalizacji - LinearDecay itp. 
 
 
\begin{figure}[H]
\centering
\includegraphics[width=\linewidth]{06-implementacja/rys/normalizer-hierarchy.PNG}
\caption{Hierarchia klas: normalizacja}
\label{fig:criterion-classes}
\end{figure}

\subsubsection{Metoda ważonych kryteriów}

Metoda ważonych kryteriów polega na opisaniu funkcji oceny decyzji jako sumy ważonej ocen poszczególnych kryteriów [?]. Konieczne jest przyporządkowanie wagi dla każdego z kryteriów. Ocena poszczególnych decyzji obliczana jest według wzoru: 

\begin{equation}\label{eq:mwk}
F(x)=\sum_{i=1}^{k}w_{i}f_{i}(x)
\end{equation}

gdzie k - ilość kryteriów, x - wektor rozwiązań, $w_{i}$ - wagi takie że:

\begin{equation*}
w \in [0, 1] \mbox{ oraz } \sum_{i=1}^{k}w_{i} = 1
\end{equation*}

Metoda ta została wybrana ze względu na możliwość dynamicznego doboru wag poszczególnych kryteriów. Niektóre z tych wag będą dobierane przez kapitana zgodnie z preferencjami jego drużyny. 


\begin{minipage}{\linewidth}
\begin{lstlisting}[label=list:resource, caption=Przykładowa rejestracja kontrolera, basicstyle=\footnotesize\ttfamily]

@Service
public class WeightedCriteriaAggregator {

    public double aggregate(List<WeightedCriteria> weightedCriteria) {
      return aggregate(weightedCriteria.stream());
    }

    public double aggregate(Stream<WeightedCriteria> stream) {
      return stream
      .mapToDouble(
       crit -> crit.getWeight() * crit.getCriteria().getNormalizedValue()
       )
      .sum();
    }

}

\end{lstlisting}
\end{minipage}

\section{Implementacja aplikacji klienckiej}

\subsection{Modularność}

Modularyzacja + podział na komponenty

Lorem ipsum dolor sit amet, consectetur adipiscing elit, sed do eiusmod tempor incididunt ut labore et dolore magna aliqua. Ut enim ad minim veniam, quis nostrud exercitation ullamco laboris nisi ut aliquip ex ea commodo consequat. Duis aute irure dolor in reprehenderit in voluptate velit esse cillum dolore eu fugiat nulla pariatur. Excepteur sint occaecat cupidatat non proident, sunt in culpa qui officia deserunt mollit anim id est laborum. Duis aute irure dolor in reprehenderit in voluptate velit esse cillum dolore eu fugiat nulla pariatur. Excepteur sint occaecat cupidatat non proident, sunt in culpa qui officia deserunt mollit anim id est laborum. 


\begin{figure}[H]
\centering
\includegraphics[width=\linewidth]{06-implementacja/rys/modularyzacja.PNG}
\caption{Hierarchia klas: normalizacja}
\label{fig:criterion-classes}
\end{figure}


\subsection{Stan aplikacji}

Stan aplikacji
Zarządzanie stanem aplikacji Obsługa stanu aplikacji 

Lorem ipsum dolor sit amet, consectetur adipiscing elit, sed do eiusmod tempor incididunt ut labore et dolore magna aliqua. Ut enim ad minim veniam, quis nostrud exercitation ullamco laboris nisi ut aliquip ex ea commodo consequat. Duis aute irure dolor in reprehenderit in voluptate velit esse cillum dolore eu fugiat nulla pariatur. Excepteur sint occaecat cupidatat non proident, sunt in culpa qui officia deserunt mollit anim id est laborum. Lorem ipsum dolor sit amet, consectetur adipiscing elit, sed do eiusmod tempor incididunt ut labore et dolore magna aliqua. Ut enim ad minim veniam, quis nostrud exercitation ullamco laboris nisi ut aliquip ex ea commodo consequat. Duis aute irure dolor in reprehenderit in voluptate velit esse cillum dolore eu fugiat nulla pariatur. Excepteur sint occaecat cupidatat non proident, sunt in culpa qui officia deserunt mollit anim id est laborum

\begin{figure}[H]
\centering
\includegraphics[width=\linewidth]{06-implementacja/rys/modularyzacja.PNG}
\caption{Hierarchia klas: normalizacja}
\label{fig:criterion-classes}
\end{figure}

\subsection{Formularze}

formularze, Walidacja formularzy, pare screenow, odowlanie do instrukcji uzytkownika w zalaczniku

Lorem ipsum dolor sit amet, consectetur adipiscing elit, sed do eiusmod tempor incididunt ut labore et dolore magna aliqua. Ut enim ad minim veniam, quis nostrud exercitation ullamco laboris nisi ut aliquip ex ea commodo consequat. Duis aute irure dolor in reprehenderit in voluptate velit esse cillum dolore eu fugiat nulla pariatur. Excepteur sint occaecat cupidatat non proident, sunt in culpa qui officia deserunt mollit anim id est laborum. Lorem ipsum dolor sit amet, consectetur adipiscing elit, sed do eiusmod tempor incididunt ut labore et dolore magna aliqua. Ut enim ad minim veniam, quis nostrud exercitation ullamco laboris nisi ut aliquip ex ea commodo consequat. Duis aute irure dolor in reprehenderit in voluptate velit esse cillum dolore eu fugiat nulla pariatur. Excepteur sint occaecat cupidatat non proident, sunt in culpa qui officia deserunt mollit anim id est laborum

\begin{figure}[H]
\centering
\includegraphics[width=0.7\linewidth]{06-implementacja/rys/modularyzacja.PNG}
\caption{Hierarchia klas: normalizacja}
\label{fig:criterion-classes}
\end{figure}

\section{Bezpieczeństwo}

Szczególną uwagę poświęcono implementacji zabezpieczeń systemu oraz zgromadzonych danych użytkowników. Funkcjonalności związane z bezpieczeństwem zostały zrealizowane bazując na module Spring Security, który dostarcza wiele przydatnych mechanizmów oraz możliwości ich konfiguracji.

Hasła użytkowników przechowywane są w formie zaszyfrowanej za pomocą funkcji skrótu BCrypt. Spring Security dostarcza klasę implementującą ten algorytm - BCryptPasswordEncoder. Główną zaletą takiego sposobu szyfrowania jest generowanie losowego ciągu znaków, tak zwanej soli, który dodatkowo wzmacnia bezpieczeństwo hasła. Zastosowanie soli przy hashowaniu znacznie utrudnia złamanie hasła przez ataki przy użyciu tęczowych tabel. W przypadku algorytmu bcrypt wygenerowana sól jest przechowywana razem z zaszyfrowanym hasłem w bazie danych. W przypadku prób logowania jest ona wyciągana z bazy oraz używana do przetworzenia podanego hasła w celu porównania zgodności.


Autoryzacja zapytań (JWT)


\begin{figure}[H]
\centering
\includegraphics[width=0.7\linewidth]{06-implementacja/rys/modularyzacja.PNG}
\caption{Hierarchia klas: normalizacja}
\label{fig:criterion-classes}
\end{figure}