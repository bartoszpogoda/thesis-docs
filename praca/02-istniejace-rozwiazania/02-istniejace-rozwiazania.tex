\chapter{Istniejące rozwiązania}

W niniejszym rozdziale przedstawione zostaną wybrane z istniejących rozwiązań wspierających komunikację pomiędzy zawodnikami. Dla poszczególnych platform zostaną wyszczególnione ich główne założenia oraz funkcjonalności.


\section{Facebook}

Facebook jest serwisem społecznościowym zrzeszającym prawie 2 miliardy osób z całego świata <odw wikipedia?>. Główną misją Facebooka jest zbliżanie do siebie ludzi poprzez umożliwianie budowy społeczności.

Społeczność osób uprawiających sporty zespołowe nie stanowi tutaj wyjątku. Na Facebooku istnieje bardzo dużo grup tematycznych, których celem jest gromadzenie osób uprawiających pewną dyscyplinę sportu w określonym regionie. Osoby szukające osób do gry bardzo często tworzą posty podając takie informacje jak miejsce oraz termin spotkania, preferowany wiek oraz poziom umiejętności. Chętne osoby zgłaszają się pod postem lub poprzez wiadomość prywatną. 

Szukanie osób do gry poprzez portal Facebook jest bardzo często wybieranym rozwiązaniem głownie ze względu na dużą ilość użytkowników oraz szeroką dostępność serwisu. W tabeli XX przedstawiono zestawienie wybranych z publicznych grup w mieście Wrocław


\begin{table}[htb]
\centering\small
\caption{Przykładowe grupy dla zawodników na Facebook - stan z dnia 20.11.2018r}
\label{tab:szablon}
\begin{tabularx}{\linewidth}{|p{.55\linewidth}|X|}\hline
Nazwa grupy & Liczba członków \\ \hline\hline
Piłka nożna Wrocław & 4689  \\ \hline
Siatkówka Wrocław & 4460  \\ \hline
Koszykówka Wrocław & 1686 \\ \hline
Piłka nożna Wrocław - dla PWR & 273  \\ \hline
\end{tabularx}
\end{table}

>Coś tutaj więcej...Może o jakichś wadach takiego zastosowania

\section{Playarena.pl}

Playarena jest portalem skierowanym do drużyn piłki nożnej 6 osobowej. 

O playarena (http://playarena.pl/corobimy)

\section{SportsMatchMaker.com.au}

O Australinskim Sportsmatchmaker (http://www.sportsmatchmaker.com.au/aboutus.aspx)
