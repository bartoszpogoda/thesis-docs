\chapter{Walidacja}

Czy weryfikacja środowiskowa?

Coś ogólnie o tym że ważna jest weryfikacja a w systemach webowych ważne jest badanie użyteczności interfejsu

\section{Metodyka badań}

System został dwukrotnie poddany weryfikacji środowiskowej. Pierwsze badania odbyły się po ukończeniu implementacji prototypu systemu. W celu umożliwienia weryfikacji do systemu zostały wprowadzone przykładowe (zmyślone) dane zawodników, drużyn oraz obiekty sportowe.

Badania odbyły się w formie badań moderowanych, czyli z udziałem osoby nadzorującej ich przebieg. Moderatorem w przypadku wszystkich badań był autor tej pracy. Dla uczestników badania została przygotowana ankieta składająca się z trzech części. Pierwsza część była ankietą wstępną uzupełnianą przed badaniami. (coś żeby dowiedzieć się ile ma lat itp). Kolejne dwie części zawierały opisy zadań przygotowanych dla użytkowników oraz pytania kontrolne dotyczące intuicyjności poszczególnych procesów. Ze względu na obecność moderatora osoby uczestniczące w badaniu zostały poproszone o głośne myślenie oraz wyrażanie uwag. Moderator podczas badań na bieżąco notował istotne akcje podejmowane przez użytkowników oraz ich uwagi.

Do badań zostały wybrane osoby potencjalnie zainteresowane tematem, czyli osoby uprawiające sporty zespołowe. W celu zbadania użyteczności systemu dla różnych grup wiekowych zaproszono osoby w przedziale od 15 do 35 lat. Wśród ankietowanych były osoby profesjonalnie zajmujące się systemami webowymi jak również osoby spoza tej branży.

\section{Wyniki badań}

