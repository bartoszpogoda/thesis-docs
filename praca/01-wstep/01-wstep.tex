\chapter{Wstęp}
\section{Wprowadzenie}

Sport jest dziedziną towarzyszącą ludzkości od najdawniejszych czasów. Sprawność fizyczna była niezwykle ważną cechą już dla ludzi pierwotnych, dla których niejednokrotnie mogła ona być czynnikiem decydującym o przetrwaniu. W dalszych dziejach ludzkości duży wpływ na narodziny oraz rozwój kultury fizycznej miały starożytne państwa, które organizowały Igrzyska Sportowe.

Ewolucja sportu trwa nadal. Badania wykazują duży wpływ aktywnego trybu życia na zdrowie fizyczne oraz psychiczne człowieka. Aktywność fizyczna jest nieustannie promowana przez lekarzy oraz inne instytucje. Ogromny wpływ na popularyzację zdrowego trybu życia mają również wszelkie organizowane zawody sportowe, które są bardzo popularne w mediach.

Szczególną dziedziną sportu są sporty zespołowe, w których po za indywidualnymi zdolnościami zawodników, ogromne znaczenie ma współpraca. Umiejętność współdziałania w celu osiągnięcia wspólnego celu jest niezwykle istotną cechą, przydatną w wielu życiowych sytuacjach. Wiele sportów zespołowych swój fenomen opiera również na rywalizacji, która daje zawodnikom dodatkową motywację do samorozwoju.

Sporty zespołowe są od wielu lat promowane poprzez organizację dużych imprez takich jak Mistrzostwa Świata, Mistrzostwa Europy. Wraz z popularnością sportów zespołowych rośnie liczba osób, które uprawiają sporty zespołowe amatorsko\footnote{Amatorsko czyli traktując sport jako hobby, dodatek do życia, a nie jego główny kierunek i źródło utrzymania}. Osoby takie poprzez wspólną grę oraz rywalizację mogą poprawić swoją sprawność fizyczną aktywnie spędzając czas, jak również nawiązać nowe znajomości.

Popularyzacja aktywnego trybu życia w społeczeństwie stanowi motywację oraz uzasadnienie zapotrzebowania na rozwiązania, które przy użyciu technologii dostępnych w dzisiejszych czasach, wspierałyby komunikację pomiędzy osobami uprawiającymi amatorsko sporty zespołowe. 

W celu zapewnienia dużej dostępności rozwiązania bez względu na sprzęt oraz położenie użytkownika naturalnym wyborem jest umieszczenie platformy w Internecie, do którego dostęp ma znaczna większość populacji. Istniejące platformy internetowe takie jak Facebook, YouTube sukcesywnie wspomagają budowanie społeczności ludzi o wspólnych zainteresowaniach, w dużej mierze ze względu na swoją szeroką dostępność.

\section{Cel i zakres pracy}

Celem niniejszej pracy jest zaprojektowanie oraz implementacja prototypu platformy internetowej, która zrzeszałaby osoby uprawiające sporty zespołowe w sposób amatorski poprzez:

\begin{itemize}
  \item utrzymywanie bazy zawodników,
  \item utrzymywanie bazy drużyn,
  \item utrzymywanie bazy obiektów sportowych,
  \item wspomaganie poszukiwania rywali,
  \item wspomaganie umawiania się na rozgrywkę.
\end{itemize} 

Do zakresu pracy należy projekt ogólnego rozwiązania, które mogłoby z powodzeniem być zastosowane dla różnych dyscyplin sportów zespołowych. Implementacja części klienckiej zostanie jednak zawężona do jednej dyscypliny sportu zespołowego - koszykówki 3 na 3. Wybór padł na tę dyscyplinę ze względu na jej rosnącą popularność oraz brak istniejących rozwiązań ukierunkowanych na organizację rozgrywek w tej dyscyplinie.

Projekt został nazwany Team Challenge. Nazwa ta nawiązuje do głównej funkcjonalności projektowanego systemu, czyli do wspierania rywalizacji pomiędzy drużynami poprzez rzucanie wyzwań.

\begin{comment}


Jako dyscyplina została wybrana koszykówka 3 na 3, która jest młodą . Projektowany system został nazwany Team Challenge. Nazwa ta nawiązuje do głównej funkcjonalności systemu jaką jest rzucanie wyzwań. "Team" odnosi się do głównej grupy docelowej systemu czyli drużyn. "Challenge" nawiązuje do funkcjonalności rzucania wyzwań. I coś że rózne formy wyzwań są popularne. IceBucket Challenge itp. I że może budzić z tego powodu zainteresowanie.

Projektowany system został nazwany Team Challenge. 

Ze względu na duży rozwój

TODO O tym że projekt ogólny a implementacja dla wybranej dziedziny a konkretnie koszykówki 3 na 3 która budzi co raz większe zainteresowanie i np będzie na igrzyskach olimpijskich. Wybór ze względu na popularność dyscypliny i brak dla niej istniejącego rozwiązania\cite{JS07}).
\end{comment}


\section{Układ pracy}

Poniżej wymieniono kolejne rozdziały pracy wraz z krótkimi opisami ich zawartości.

\begin{comment}
W kolejnym rozdziale niniejszej pracy przedstawiono wybrane z istniejących platform (?). W trzecim rozdziale zawarto koncept rozwiązania. Czwarty rozdział obejmuje projekt systemu. Piąty rozdział przybliża architekturę systemu oraz technologie wybrane do jego budowy. W następnym, szóstym, rozdziale

O układzie pracy, przejście od przeglądu rozwiązań poprzez koncept, projekt techniczny aż do szczegółów implementacji oraz jej walidacji środowiskowej.

Następny rozdział niniejszej pracy zawiera przegląd istniejących platform, które realizują podobne cele do projektowanego systemu. Kolejne rozdziały dotyczą projektowanego systemu. W trzecim rozdziale zawarto projekt konceptualny. Czwarty rozdział obejmuje projekt techniczny systemu. W piątym rozdziale przybliżono 

\end{comment}

\begin{itemize}

\item \textbf{Istniejące rozwiązania} - Przegląd istniejących platform, które realizują cele zbliżone do projektowanego systemu.  \\

\item \textbf{Koncepcja systemu Team Challenge} - Przedstawienie ogólnej koncepcji systemu. Przybliżenie sposobów w jaki system będzie realizował wymienione wyżej cele.   \\

\item \textbf{Projekt systemu Team Challenge} - Techniczny projekt systemu zrealizowany zgodnie ze standardami języka UML oraz uzupełniony o diagram związków encji. \\

\item \textbf{Architektura i technologie} - Architektura systemu oraz przegląd wykorzystanych technologii wraz z motywacjami ich użycia.  \\

\item \textbf{Implementacja i działanie systemu Team Challenge} - Rozdział, w którym zawarto opis przebiegu implementacji oraz wybrane szczegóły działania systemu.   \\

\item \textbf{Ocena użyteczności} - Rozdział poświęcony testom użyteczności, którym zostało poddane zaimplementowane rozwiązanie.  Przedstawienie metodyki badań oraz ich wyników. \\

\item \textbf{Perspektywy rozwoju} - Nakreślenie pomysłów na dalsze kierunki prac nad systemem.  \\

\item \textbf{Podsumowanie} - Ostatni rozdział pracy, zawierający przemyślenia na temat projektu oraz wnioski wysnute podczas jego realizacji.  \\

\end{itemize}


