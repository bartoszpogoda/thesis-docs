\chapter{Wstęp}
\section{Wprowadzenie}

Sport jest aspektem towarzyszącym ludzkości od najdawniejszych czasów. Sprawność fizyczna była niezwykle ważną cechą już dla ludzi pierwotnych, dla których niejednokrotnie mogła ona być czynnikiem decydującym o przetrwaniu. W dalszych dziejach ludzkości duży wpływ na narodziny oraz rozwój kultury fizycznej miały starożytne państwa, które organizowały Igrzyska Sportowe.

Ewolucja sportu trwa nadal. Badania wykazują duży wpływ aktywnego trybu życia na zdrowie fizyczne oraz psychiczne człowieka. Aktywność fizyczna jest nieustannie promowana przez lekarzy oraz inne instytucje. Ogromny wpływ na popularyzację zdrowego trybu życia mają również wszelkie organizowane zawody sportowe, które są bardzo popularne w mediach.

Szczególną dziedziną sportu są sporty zespołowe, w których po za indywidualnymi zdolnościami zawodników, ogromne znaczenie ma współpraca. Umiejętność współdziałania w celu osiągnięcia wspólnego celu jest niezwykle istotną cechą, przydatną w wielu życiowych sytuacjach. Wiele sportów zespołowych swój fenomen opiera również na rywalizacji, która daje zawodnikom dodatkową motywację do samorozwoju.

Sporty zespołowe są od wielu lat promowane poprzez organizację dużych imprez takich jak Mistrzostwa Świata, Mistrzostwa Europy. Wraz z popularnością sportów zespołowych rośnie liczba osób, które uprawiają sporty zespołowe amatorsko\footnote{Amatorsko czyli traktując sport jako hobby, dodatek do życia, a nie jego główny kierunek i źródło utrzymania}. Osoby takie poprzez wspólną grę oraz rywalizację mogą poprawić swoją sprawność fizyczną aktywnie spędzając czas, jak również nawiązać nowe znajomości.

Popularyzacja aktywnego trybu życia w społeczeństwie stanowi motywację oraz uzasadnienie zapotrzebowania na rozwiązania, które przy użyciu technologii dostępnych w dzisiejszych czasach, wspierałyby komunikację pomiędzy osobami uprawiającymi amatorsko sporty zespołowe. 

W celu zapewnienia dużej dostępności rozwiązania bez względu na sprzęt oraz położenie użytkownika naturalnym wyborem jest umieszczenie platformy w Internecie, do którego dostęp ma znaczna większość populacji. Istniejące platformy internetowe takie jak Facebook, YouTube sukcesywnie wspomagają budowanie społeczności ludzi o wspólnych zainteresowaniach, w dużej mierze ze względu na swoją szeroką dostępność.

\section{Cel i zakres pracy}

Celem niniejszej pracy jest zaprojektowanie oraz implementacja prototypu platformy internetowej, która zrzeszałaby osoby uprawiające sporty zespołowe w sposób amatorski poprzez:

\begin{itemize}
  \item utrzymywanie bazy zawodników,
  \item utrzymywanie bazy drużyn,
  \item utrzymywanie bazy obiektów sportowych,
  \item wspomaganie poszukiwania rywali,
  \item wspomaganie umawiania się na rozgrywkę.
\end{itemize} 

Do zakresu pracy należy projekt ogólnego rozwiązania, które mogłoby z powodzeniem być zastosowane dla różnych dyscyplin sportów zespołowych. Implementacja zostanie jednak zrealizowana dla wybranej dyscypliny.

\begin{comment}
Ze względu na duży rozwój

TODO O tym że projekt ogólny a implementacja dla wybranej dziedziny a konkretnie koszykówki 3 na 3 która budzi co raz większe zainteresowanie i np będzie na igrzyskach olimpijskich. Wybór ze względu na popularność dyscypliny i brak dla niej istniejącego rozwiązania\cite{JS07}).
\end{comment}


\section{Układ pracy}

W kolejnym rozdziale niniejszej pracy przedstawiono wybrane z istniejących platform (?). W trzecim rozdziale zawarto koncept rozwiązania. Czwarty rozdział obejmuje projekt systemu. Piąty rozdział przybliża architekturę systemu oraz technologie wybrane do jego budowy. W następnym, szóstym, rozdziale

O układzie pracy, przejście od przeglądu rozwiązań poprzez koncept, projekt techniczny aż do szczegółów implementacji oraz jej walidacji środowiskowej.

\begin{enumerate}
\item xd
\end{enumerate}