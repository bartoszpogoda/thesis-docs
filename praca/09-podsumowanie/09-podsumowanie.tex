\chapter{Podsumowanie}

Przeprowadzenie projektu składającego się z etapu koncepcyjnego, następnie projektowego, implementacyjnego oraz weryfikacyjnego, pozwoliło na uzyskanie zadowalającego efektu końcowego, w postaci zaawansowanego prototypu systemu zrzeszającego sportowców. Podczas poszczególnych etapów projektu korzystano ze sprawdzonych standardów, technologii oraz metodyk działania. Miało to duży wpływ na sprawny rozwój projektu. W przypadku napotkanych problemów oraz niejasności bardzo przydatna okazywała się literatura w postaci dokumentacji oraz artykułów naukowych.

Przeprowadzone testy użyteczności rozwiązania pozwoliły na zidentyfikowanie problemów związanych z intuicyjnością interfejsu oraz wprowadzenie usprawnień. Konfrontacja systemu we wczesnym etapie rozwoju z potencjalnymi użytkownikami pozwoliła również na lepsze zrozumienie ich oczekiwań oraz przystosowanie kierunku rozwoju platformy.

Uzyskany produkt końcowy może stanowić wartą rozpatrzenia alternatywę dla istniejących rozwiązań. Głównym elementem wyróżniającym system Team Challenge spośród innych dostępnych propozycji jest nowatorskie wspomaganie poszukiwania rywali, które zostało zrealizowane w oparciu o zagadnienie optymalizacji wielokryterialnej. Algorytm został zrealizowany w sposób w pełni otwarty na rozszerzenia, co umożliwia jego profilowanie z uwzględnieniem rzeczywistych potrzeb grupy docelowej.

Praca nad platformą była dla autora niniejszej pracy ciekawym doświadczeniem. Wynikało to z dużego zainteresowanie tematem sportów zespołowych oraz chęcią rozwoju w obrębie systemów internetowych. Rozwiązywanie napotkanych problemów pozwoliło na poszerzenie wiedzy na temat wykorzystywanych technologii oraz wiedzy ogólnej dotyczącej inżynierii systemów informatycznych.






