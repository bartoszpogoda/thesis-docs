\chapter{Perspektywy rozwoju}

Podczas trwania prac nad projektem narodziło się wiele pomysłów mogących urozmaicić działanie platformy. W niniejszym rozdziale zostały przedstawione wybrane kierunki, które mogłyby wpłynąć na atrakcyjność oraz jakość usług dostarczanych przez system.

\section{Kontynuacja testów użyteczności}

W ramach dalszego rozwoju systemu mogłaby zostać podjęta kontynuacja testów walidacyjnych. Dotychczas przeprowadzone badania moderowane sprawdzały intuicyjność poszczególnych procesów, a kryterium oceny była poprawność podjętych przez użytkownika działań. Kolejny etap testów mógłby dodatkowo sprawdzać ile czasu zajmuje użytkownikom wykonanie poszczególnych akcji. Analiza wyników czasowych mogłaby posłużyć identyfikacji obszarów, które wymagają przebudowy w celu przyśpieszenia interakcji.


\section{Aplikacja mobilna}

Naturalnym kierunkiem rozwoju w przypadku platformy internetowej jest dostarczenie użytkownikom aplikacji mobilnej. Rozwiązanie takie ułatwiłoby użytkownikom dostęp do systemu z różnych lokalizacji. Aplikacja również mogłaby za pomocą powiadomień informować użytkowników o nowych zdarzeniach dotyczących umawianych wyzwań.

\section{Partnerskie obiekty sportowe}

Kolejną perspektywą jest nawiązanie współpracy z partnerami w postaci zarządców obiektów sportowych z ograniczonym dostępem. Ideą współpracy byłaby reklama obiektu sportowego w zamian za udostępnienie obiektu w określonych godzinach dla drużyn \textit{Team Challenge}. Do systemu mógłby zostać wprowadzony nowy typ obiektów sportowych - obiekty partnerskie, wymagające uprzedniej rezerwacji. Należałoby również ograniczać dostęp do takich obiektów, na przykład poprzez wprowadzenie wirtualnej wewnątrz-systemowej waluty w postaci tokenów, którymi drużyny płaciłyby za rezerwację obiektów. Drużyny mogłyby otrzymywać tokeny co jakiś czas oraz za zakończone wyzwania.

\section{System rankingowy}

Interesującym kierunkiem rozwoju jest zdefiniowanie systemu rankingowego drużyn. Wyniki rozgrywek pomiędzy drużynami mogłyby wpływać na pozycję rankingową drużyny. Pozycja rankingowa drużyny mogłaby stanowić nowe kryterium dla algorytmu podczas poszukiwania rywali. Przykładowym systemem rankingowym, który mógłby znaleźć tutaj zastosowanie jest system \textit{ELO}, używany w rozgrywkach szachowych oraz wielu grach komputerowych online. Rywalizacja o miejsca rankingowe mogłaby stanowić dodatkową motywację do rozwoju drużyn. 

\section{Zwiększenie możliwości zawodników}

W zaimplementowanym systemie większość akcji dotyczących poszukiwania przeciwników, negocjacji spotkania, wprowadzania wyników oraz ocen rywali odbywa się za pośrednictwem kapitana. Po założeniu profilu zawodnika oraz dołączeniu do drużyny funkcjonalności zawodników, nie będących kapitanami, ograniczają się do przeglądania społeczności, przeglądania wyzwań oraz wprowadzania obiektów sportowych. Wpływ zawodników na poszczególne funkcjonalności mógłby zostać zwiększony poprzez:

\begin{itemize}
    \item umożliwienie zawodnikom szukania rywali oraz proponowania spotkań do akceptacji przez kapitanów,
    \item umożliwienie zawodnikom głosowania na poszczególne oferty czasu oraz miejsca spotkania wprowadzane przez kapitanów.
\end{itemize}

\section{Czat}

W celu usprawnienia komunikacji pomiędzy drużynami umawiającymi się na mecz mógłby zostać wprowadzony dodatkowy system wymiany informacji w formie czatu. Luźna wymiana wiadomości mogłaby również posłużyć wstępnemu zapoznaniu się drużyn, a co za tym idzie zwiększeniu komfortu oraz pewności pierwszego spotkania.

% \section{System notyfikacji}

% Duży wpływ na jakość usług dostarczanych przez \textit{Team Challenge} byłoby zaprojektowanie, zaimplementowanie oraz wdrożenie systemu notyfikacji. Zawodnicy mogliby być informowani o ważnych zdarzeniach dotyczących umawianych spotkań drogą mailową lub za pomocą powiadomień \textit{push}. 



\begin{comment}

proponowanei spotkan przez zawodnikow
propozycje pobliskich obiektow przy negocjacjach miejsca
system notyfikacji

\end{comment}