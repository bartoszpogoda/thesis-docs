\chapter{Perspektywy rozwoju}

Podczas trwania prac nad projektem narodziło się wiele pomysłów mogących urozmaicić działanie platformy. W niniejszym rozdziale zostaną przedstawione wybrane kierunki, które mogłyby wpłynąć na atrakcyjność oraz jakość systemu.

\section{Dalszy rozwój interfejsu użytkownika}

O pomyśle na dalsze badania itp.

\section{Aplikacja mobilna}

Naturalnym kierunkiem rozwoju w przypadku platformy internetowej jest dostarczenie użytkownikom aplikacji mobilnej. Rozwiązanie takie ułatwiłoby użytkownikom dostęp do systemu z różnych lokalizacji. Aplikacja również mogłaby za pomocą powiadomień informować użytkowników o nowych zdarzeniach dotyczących umawianych wyzwań.

\section{Partnerskie obiekty sportowe}

Kolejną perspektywą jest nawiązanie współpracy z partnerami w postaci zarządców obiektów sportowych z ograniczonym dostępem. Ideą współpracy byłaby reklama obiektu sportowego w zamian za udostępnienie obiektu w określonych godzinach dla drużyn Team Challenge. Do systemu mógłby zostać wprowadzony nowy typ obiektów sportowych - obiekty partnerskie, wymagające uprzedniej rezerwacji. Należałoby również ograniczać dostęp do takich obiektów, na przykład poprzez wprowadzenie wirtualnej wewnątrz-systemowej waluty w postaci tokenów, którymi drużyny płaciłyby za rezerwację obiektów. Drużyny mogłyby otrzymywać tokeny co jakiś czas oraz za zakończone wyzwania.

\section{System rankingowy}

Interesującym kierunkiem rozwoju jest zdefiniowanie systemu rankingowego drużyn. Wyniki rozgrywek pomiędzy drużynami mogłyby wpływać na pozycję rankingową drużyny. Pozycja rankingowa drużyny mogłaby stanowić nowe kryterium dla algorytmu podczas poszukiwania rywali. Przykładowym systemem rankingowym, który mógłby znaleźć tutaj zastosowanie jest system ELO, używany w rozgrywkach szachowych oraz wielu grach komputerowych online.

\begin{comment}

proponowanei spotkan przez zawodnikow
propozycje pobliskich obiektow przy negocjacjach miejsca
system notyfikacji

\end{comment}